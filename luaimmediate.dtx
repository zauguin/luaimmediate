% \iffalse meta-comment
%
%% Copyright (C) 2020 by Marcel Krueger
%%
%% This file may be distributed and/or modified under the
%% conditions of the LaTeX Project Public License, either
%% version 1.3c of this license or (at your option) any later
%% version. The latest version of this license is in:
%%
%% http://www.latex-project.org/lppl.txt
%%
%% and version 1.3 or later is part of all distributions of
%% LaTeX version 2005/12/01 or later.
%
%<*batch>
%<*gobble>
\ifx\jobname\relax\let\documentclass\undefined\fi
\ifx\documentclass\undefined
\csname fi\endcsname
%</gobble>
\input docstrip.tex
\keepsilent
\let\MetaPrefix\relax
\preamble
\endpreamble
\postamble
\endpostamble
\def\MetaPrefix{--}
\generate{%
  \file{luaimmediate.lua}{\from{luaimmediate.dtx}{luacode}}%
}
\let\MetaPrefix\DoubleperCent
\generate{%
  \file{luaimmediate.sty}{\from{luaimmediate.dtx}{package}}
  \file{luaimmediate.tex}{\from{luaimmediate.dtx}{tex-package}}}
\endbatchfile
%</batch>
%<*gobble>
\fi
\expandafter\ifx\csname @currname\endcsname\empty
\csname fi\endcsname
%</gobble>
%<*driver>
\RequirePackage{luaimmediate}
\documentclass{article}
\usepackage{csquotes,doc,metalogo,hyperref,luacolor,tikzducks,pict2e}
\RecordChanges
\MakeShortVerb\|
\begin{document}
\DocInput{luaimmediate.dtx}
\PrintIndex
\PrintChanges
\end{document}
%</driver>
%<*gobble>
\fi
%</gobble>
% \fi
%
% \GetFileInfo{luaimmediate.sty}
% \title{The luaimmediate package\thanks{This document
%        corresponds to luaimmediate~\fileversion, dated~\filedate.}}
% \author{Marcel Kr\"uger \\ \href{mailto:tex@2krueger.de}{tex@2krueger.de}}
%
% \maketitle
% Overwrite \texttt{\string\immediate} with a Lua wrapper which allows commands defined by
% \texttt{\string\luadef} to react to the presence of \texttt{\string\immediate}.
%
% If you have a command \texttt{\string\example} defined by \texttt{\string\luadef\string\example42},
% then you can allow it to appear after \texttt{\string\immediate} by writing
% \begin{verbatim}
% \directlua {require'luaimmediate'('example', true)}
% \end{verbatim}
% Then the first argument to the corresponding Lua function will be \texttt{"immediate"}
% if the command is used after \texttt{\string\immediate}.
% By passing a function instead of \texttt{true}, an independent replacement function can
% be used instead specifically for use in an \texttt{\string\immediate} context.
% 
% \StopEventually{}
% \section{The implementation}
% \changes{0.0.1}{2020-04-09}{Initial release}
% \subsection{Lua module}
% Now we can define our main Lua module:
% \iffalse
%<*gobble>
\RequirePackage{docstrip-luacode}
\begin{docstrip-luacode}{luaimmediate}
%</gobble>
%<*luacode>
% \fi
%    \begin{macrocode}
local get_command = token.get_command
local get_mode = token.get_mode
local is_defined = token.is_defined
local is_token = token.is_token
local put_next = token.put_next
local scan_token = token.scan_token
local token_create = token.create
local token_new = token.new

local extension_cmd = token.command_id'extension'
local lua_call_cmd = token.command_id'lua_call'
local lua_function_call_cmd = token.command_id'lua_function_call'
local immediate_code = 4
local immediate_token = token.new(immediate_code, extension_cmd)
local minus_token = token.new(0x2D, 12)

local tointeger = math.tointeger
local error = error

local immediate_lookup = {}

local function new_immediate()
  local token = scan_token()
  local tok = (get_command(token) << 21) | get_mode(token)
  immediate_function = immediate_lookup[tok]
  if immediate_function then
    return immediate_function()
  else
    put_next(immediate_token, token)
  end
end
local functions = lua.get_functions_table()
local immediate_func = luatexbase
                         and luatexbase.new_luafunction'immediate'
                         or (string.unpack(">I", 'imm') & 0x1FFFFF)
functions[immediate_func] = new_immediate
token.set_lua('immediate', immediate_func, 'protected', 'global')
%    \end{macrocode}
% The next technically is not necessary, but it does not hurt either.
%    \begin{macrocode}
immediate_lookup[(lua_call_cmd << 21) | immediate_func]
    = new_immediate

return function(id, func)
  if not is_token(id) then
    local tid = type(id)
    if tid == 'string' then
      if not is_defined(id) then
        error('Attempt to define an immediate meaning to an \z
          undefined csname.')
      end
      id = token_create(id)
    elseif tid == 'number' then
      id = tointeger(id)
      if (not id) or id <= 0 then
        error('Invalid number passed to luaimmediate.')
      end
      id = token_new(id, lua_call_cmd)
    else
      error('Invalid argument passed to luaimmediate.')
    end
  end
  local idtok = (get_command(id) << 21) | get_mode(id)
  if func == nil then
    return immediate_lookup[idtok]
  elseif func == true then
    if get_command(id) ~= lua_call_cmd then
      error[[Explicit function required for weird immediate tokens.]]
    end
%    \end{macrocode}
% Now we cache |i|, not |functions[i]|. Therefore we pick up changes to
% |functions[i]| at some later point. That is by design, if you do want
% to avoid it, just pass a function instead.
%    \begin{macrocode}
    local i = get_mode(id)
    func = function()
      return functions[i]'immediate'
    end
  end
  immediate_lookup[idtok] = func or nil
end
%    \end{macrocode}
% \iffalse
%</luacode>
%<*gobble>
\end{docstrip-luacode}
%</gobble>
% \fi
% \subsection{\TeX\ support package}
% Most of the time this \TeX\ helper is not needed, but sometimes it can be useful to ensure
% |\immediate| is redefined early enough.
% \iffalse
%<*package|tex-package>
% \fi
%    \begin{macrocode}
%<*package>
\NeedsTeXFormat{LaTeX2e}
\ProvidesPackage
  {luaimmediate}
  [2020/04/09 v0.0.1 Allow Lua commands to react to immediate]
%</package>
%    \end{macrocode}
% Only \LuaLaTeX{} is supported.
% For other engines we show an error.
%    \begin{macrocode}
\ifx\directlua\undefined
%    \end{macrocode}
% \iffalse
%<*gobble>
\iffalse
%</gobble>
% \fi \fi
%    \begin{macrocode}
%<*tex-package>
  \begingroup
    \ifx\PackageError\undefined
      \def\PackageError#1#2#3{\errhelp{#3}\errmessage{#1: #2}}
    \fi
%</tex-package>
%    \end{macrocode}
% \iffalse \iffalse
%<*gobble>
\fi
%</gobble>
% \fi
%    \begin{macrocode}
    \PackageError{luaimmediate}{LuaLaTeX required}%
    {luaimmediate requires LuaLaTeX.
     Maybe you forgot to switch the engine in your editor?}
%    \end{macrocode}
% \iffalse
%<*gobble>
\iffalse
%</gobble>
% \fi \fi
%    \begin{macrocode}
%<*tex-package>
  \endgroup
%</tex-package>
%    \end{macrocode}
% \iffalse \iffalse
%<*gobble>
\fi
%</gobble>
% \fi
%    \begin{macrocode}
\fi
%    \end{macrocode}
% The actual functionality is not that interesting: We just load the
% Lua module.
%    \begin{macrocode}
\directlua{require'luaimmediate'}
\endinput
%    \end{macrocode}
% \iffalse
%</package|tex-package>
% \fi
% \Finale
